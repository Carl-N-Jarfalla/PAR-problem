\titledquestion{Vertex}

\begin{boxdef*}{Vertex}
  Vertex är den eller de punkt(er) där en kurva vänder ''håll''. Kurvan av en andragradsfunktion har en vändpunkt. Kurvor av högre grader kan ha flera.
  Ordet vertex är latin med betydelsen ''topp-punkt, vändpunkt''.
\end{boxdef*}

Denna vecka ska vi undersöka hur koefficienten för $x$ påverkar grafen till parabeln $y=3x^2 +bx-2$.

\begin{parts}
  \part Vi börjar med att betrakta positiva värden på $b$. Plotta $y=3x^2 +x-2$, $y=3x^2 +2x-2$ och $y=3x^2 +6x-2$ i samma koordinatsystem. Beskriv hur värdet på $b$ påverkar
  vertex för parabeln. Testa gärna fler värden på $b$ om du ännu inte känner dig säker på dina observationer.

  \part Plotta nu $y=3x^2 +bx-2$ för $b = -1, -2 \text{ och } -3$. När koefficienten är negativ, hur påverkar dess värde kurvans vertex?

  \part Baserat på dina svar i (a) och (b) beskriv hur värdet på $b$ påverkar positionen för vertex. Ditt svar ska gälla för alla värden på $b$, positiva, negativa och noll.

  \part Alla de plottar du gjort skär y-axeln i samma punkt. Ge en enkel förklaring varför detta sker.

\end{parts}

\titledquestion{Vertexform}[\medel]

Alla andragradsfunktioner $y=ax^2 +bx+c$ kan via kvadratkomplettering skrivas på vertexform, $y=A(x-B)^2+C$ där $A,B \text{ och } C$ är konstanter. 

Skriv om funktionen i första uppgiften på vertexform och jämför med de slutsatser om positionen av vertex du drog i första uppgiften.
